% Options for packages loaded elsewhere
\PassOptionsToPackage{unicode}{hyperref}
\PassOptionsToPackage{hyphens}{url}
\PassOptionsToPackage{dvipsnames,svgnames,x11names}{xcolor}
%
\documentclass[
  letterpaper,
  DIV=11,
  numbers=noendperiod]{scrartcl}

\usepackage{amsmath,amssymb}
\usepackage{iftex}
\ifPDFTeX
  \usepackage[T1]{fontenc}
  \usepackage[utf8]{inputenc}
  \usepackage{textcomp} % provide euro and other symbols
\else % if luatex or xetex
  \usepackage{unicode-math}
  \defaultfontfeatures{Scale=MatchLowercase}
  \defaultfontfeatures[\rmfamily]{Ligatures=TeX,Scale=1}
\fi
\usepackage{lmodern}
\ifPDFTeX\else  
    % xetex/luatex font selection
\fi
% Use upquote if available, for straight quotes in verbatim environments
\IfFileExists{upquote.sty}{\usepackage{upquote}}{}
\IfFileExists{microtype.sty}{% use microtype if available
  \usepackage[]{microtype}
  \UseMicrotypeSet[protrusion]{basicmath} % disable protrusion for tt fonts
}{}
\makeatletter
\@ifundefined{KOMAClassName}{% if non-KOMA class
  \IfFileExists{parskip.sty}{%
    \usepackage{parskip}
  }{% else
    \setlength{\parindent}{0pt}
    \setlength{\parskip}{6pt plus 2pt minus 1pt}}
}{% if KOMA class
  \KOMAoptions{parskip=half}}
\makeatother
\usepackage{xcolor}
\setlength{\emergencystretch}{3em} % prevent overfull lines
\setcounter{secnumdepth}{-\maxdimen} % remove section numbering
% Make \paragraph and \subparagraph free-standing
\makeatletter
\ifx\paragraph\undefined\else
  \let\oldparagraph\paragraph
  \renewcommand{\paragraph}{
    \@ifstar
      \xxxParagraphStar
      \xxxParagraphNoStar
  }
  \newcommand{\xxxParagraphStar}[1]{\oldparagraph*{#1}\mbox{}}
  \newcommand{\xxxParagraphNoStar}[1]{\oldparagraph{#1}\mbox{}}
\fi
\ifx\subparagraph\undefined\else
  \let\oldsubparagraph\subparagraph
  \renewcommand{\subparagraph}{
    \@ifstar
      \xxxSubParagraphStar
      \xxxSubParagraphNoStar
  }
  \newcommand{\xxxSubParagraphStar}[1]{\oldsubparagraph*{#1}\mbox{}}
  \newcommand{\xxxSubParagraphNoStar}[1]{\oldsubparagraph{#1}\mbox{}}
\fi
\makeatother

\usepackage{color}
\usepackage{fancyvrb}
\newcommand{\VerbBar}{|}
\newcommand{\VERB}{\Verb[commandchars=\\\{\}]}
\DefineVerbatimEnvironment{Highlighting}{Verbatim}{commandchars=\\\{\}}
% Add ',fontsize=\small' for more characters per line
\usepackage{framed}
\definecolor{shadecolor}{RGB}{241,243,245}
\newenvironment{Shaded}{\begin{snugshade}}{\end{snugshade}}
\newcommand{\AlertTok}[1]{\textcolor[rgb]{0.68,0.00,0.00}{#1}}
\newcommand{\AnnotationTok}[1]{\textcolor[rgb]{0.37,0.37,0.37}{#1}}
\newcommand{\AttributeTok}[1]{\textcolor[rgb]{0.40,0.45,0.13}{#1}}
\newcommand{\BaseNTok}[1]{\textcolor[rgb]{0.68,0.00,0.00}{#1}}
\newcommand{\BuiltInTok}[1]{\textcolor[rgb]{0.00,0.23,0.31}{#1}}
\newcommand{\CharTok}[1]{\textcolor[rgb]{0.13,0.47,0.30}{#1}}
\newcommand{\CommentTok}[1]{\textcolor[rgb]{0.37,0.37,0.37}{#1}}
\newcommand{\CommentVarTok}[1]{\textcolor[rgb]{0.37,0.37,0.37}{\textit{#1}}}
\newcommand{\ConstantTok}[1]{\textcolor[rgb]{0.56,0.35,0.01}{#1}}
\newcommand{\ControlFlowTok}[1]{\textcolor[rgb]{0.00,0.23,0.31}{\textbf{#1}}}
\newcommand{\DataTypeTok}[1]{\textcolor[rgb]{0.68,0.00,0.00}{#1}}
\newcommand{\DecValTok}[1]{\textcolor[rgb]{0.68,0.00,0.00}{#1}}
\newcommand{\DocumentationTok}[1]{\textcolor[rgb]{0.37,0.37,0.37}{\textit{#1}}}
\newcommand{\ErrorTok}[1]{\textcolor[rgb]{0.68,0.00,0.00}{#1}}
\newcommand{\ExtensionTok}[1]{\textcolor[rgb]{0.00,0.23,0.31}{#1}}
\newcommand{\FloatTok}[1]{\textcolor[rgb]{0.68,0.00,0.00}{#1}}
\newcommand{\FunctionTok}[1]{\textcolor[rgb]{0.28,0.35,0.67}{#1}}
\newcommand{\ImportTok}[1]{\textcolor[rgb]{0.00,0.46,0.62}{#1}}
\newcommand{\InformationTok}[1]{\textcolor[rgb]{0.37,0.37,0.37}{#1}}
\newcommand{\KeywordTok}[1]{\textcolor[rgb]{0.00,0.23,0.31}{\textbf{#1}}}
\newcommand{\NormalTok}[1]{\textcolor[rgb]{0.00,0.23,0.31}{#1}}
\newcommand{\OperatorTok}[1]{\textcolor[rgb]{0.37,0.37,0.37}{#1}}
\newcommand{\OtherTok}[1]{\textcolor[rgb]{0.00,0.23,0.31}{#1}}
\newcommand{\PreprocessorTok}[1]{\textcolor[rgb]{0.68,0.00,0.00}{#1}}
\newcommand{\RegionMarkerTok}[1]{\textcolor[rgb]{0.00,0.23,0.31}{#1}}
\newcommand{\SpecialCharTok}[1]{\textcolor[rgb]{0.37,0.37,0.37}{#1}}
\newcommand{\SpecialStringTok}[1]{\textcolor[rgb]{0.13,0.47,0.30}{#1}}
\newcommand{\StringTok}[1]{\textcolor[rgb]{0.13,0.47,0.30}{#1}}
\newcommand{\VariableTok}[1]{\textcolor[rgb]{0.07,0.07,0.07}{#1}}
\newcommand{\VerbatimStringTok}[1]{\textcolor[rgb]{0.13,0.47,0.30}{#1}}
\newcommand{\WarningTok}[1]{\textcolor[rgb]{0.37,0.37,0.37}{\textit{#1}}}

\providecommand{\tightlist}{%
  \setlength{\itemsep}{0pt}\setlength{\parskip}{0pt}}\usepackage{longtable,booktabs,array}
\usepackage{calc} % for calculating minipage widths
% Correct order of tables after \paragraph or \subparagraph
\usepackage{etoolbox}
\makeatletter
\patchcmd\longtable{\par}{\if@noskipsec\mbox{}\fi\par}{}{}
\makeatother
% Allow footnotes in longtable head/foot
\IfFileExists{footnotehyper.sty}{\usepackage{footnotehyper}}{\usepackage{footnote}}
\makesavenoteenv{longtable}
\usepackage{graphicx}
\makeatletter
\def\maxwidth{\ifdim\Gin@nat@width>\linewidth\linewidth\else\Gin@nat@width\fi}
\def\maxheight{\ifdim\Gin@nat@height>\textheight\textheight\else\Gin@nat@height\fi}
\makeatother
% Scale images if necessary, so that they will not overflow the page
% margins by default, and it is still possible to overwrite the defaults
% using explicit options in \includegraphics[width, height, ...]{}
\setkeys{Gin}{width=\maxwidth,height=\maxheight,keepaspectratio}
% Set default figure placement to htbp
\makeatletter
\def\fps@figure{htbp}
\makeatother

\KOMAoption{captions}{tableheading}
\makeatletter
\@ifpackageloaded{caption}{}{\usepackage{caption}}
\AtBeginDocument{%
\ifdefined\contentsname
  \renewcommand*\contentsname{Tabla de contenidos}
\else
  \newcommand\contentsname{Tabla de contenidos}
\fi
\ifdefined\listfigurename
  \renewcommand*\listfigurename{Listado de Figuras}
\else
  \newcommand\listfigurename{Listado de Figuras}
\fi
\ifdefined\listtablename
  \renewcommand*\listtablename{Listado de Tablas}
\else
  \newcommand\listtablename{Listado de Tablas}
\fi
\ifdefined\figurename
  \renewcommand*\figurename{Figura}
\else
  \newcommand\figurename{Figura}
\fi
\ifdefined\tablename
  \renewcommand*\tablename{Tabla}
\else
  \newcommand\tablename{Tabla}
\fi
}
\@ifpackageloaded{float}{}{\usepackage{float}}
\floatstyle{ruled}
\@ifundefined{c@chapter}{\newfloat{codelisting}{h}{lop}}{\newfloat{codelisting}{h}{lop}[chapter]}
\floatname{codelisting}{Listado}
\newcommand*\listoflistings{\listof{codelisting}{Listado de Listados}}
\makeatother
\makeatletter
\makeatother
\makeatletter
\@ifpackageloaded{caption}{}{\usepackage{caption}}
\@ifpackageloaded{subcaption}{}{\usepackage{subcaption}}
\makeatother

\ifLuaTeX
\usepackage[bidi=basic]{babel}
\else
\usepackage[bidi=default]{babel}
\fi
\babelprovide[main,import]{spanish}
% get rid of language-specific shorthands (see #6817):
\let\LanguageShortHands\languageshorthands
\def\languageshorthands#1{}
\ifLuaTeX
  \usepackage{selnolig}  % disable illegal ligatures
\fi
\usepackage{bookmark}

\IfFileExists{xurl.sty}{\usepackage{xurl}}{} % add URL line breaks if available
\urlstyle{same} % disable monospaced font for URLs
\hypersetup{
  pdftitle={Tarea 9},
  pdfauthor={Belen Raura},
  pdflang={es},
  colorlinks=true,
  linkcolor={blue},
  filecolor={Maroon},
  citecolor={Blue},
  urlcolor={Blue},
  pdfcreator={LaTeX via pandoc}}


\title{Tarea 9}
\author{Belen Raura}
\date{}

\begin{document}
\maketitle

\renewcommand*\contentsname{Tabla de Contenidos}
{
\hypersetup{linkcolor=}
\setcounter{tocdepth}{3}
\tableofcontents
}

\section{Numpy}\label{numpy}

Esta librería tiene una función para resolver un sistema de ecuaciones
mediante matrices.

\begin{Shaded}
\begin{Highlighting}[]
\ImportTok{import}\NormalTok{ numpy }\ImportTok{as}\NormalTok{ np}
\end{Highlighting}
\end{Shaded}

\section{Conjunto de ejercicios}\label{conjunto-de-ejercicios}

\subsection{Ejercicio 1}\label{ejercicio-1}

Para cada uno de los siguientes sistemas lineales, obtenga, de ser
posible, una solución con métodos gráficos. Explique los resultados
desde un punto de vista geométrico.

\subsubsection{Literal a)}\label{literal-a}

\[x_1 + 2x_2 = 0\]

\[x_1 - x_2 = 0\]

\begin{Shaded}
\begin{Highlighting}[]

\NormalTok{A }\OperatorTok{=}\NormalTok{ [}
\NormalTok{    [}\DecValTok{1}\NormalTok{, }\DecValTok{2}\NormalTok{],}
\NormalTok{    [}\DecValTok{1}\NormalTok{, }\OperatorTok{{-}}\DecValTok{1}\NormalTok{]}
\NormalTok{]}

\NormalTok{b }\OperatorTok{=}\NormalTok{ [}\DecValTok{0}\NormalTok{, }\DecValTok{0}\NormalTok{]}

\NormalTok{x }\OperatorTok{=}\NormalTok{ np.linalg.solve(A, b)}
\BuiltInTok{print}\NormalTok{(x)}
\end{Highlighting}
\end{Shaded}

\begin{verbatim}
[ 0. -0.]
\end{verbatim}

\begin{Shaded}
\begin{Highlighting}[]
\OperatorTok{\%}\NormalTok{load\_ext autoreload}
\OperatorTok{\%}\NormalTok{autoreload }\DecValTok{2}
\ImportTok{from}\NormalTok{ src }\ImportTok{import}\NormalTok{ eliminacion\_gaussiana}

\NormalTok{Ab }\OperatorTok{=}\NormalTok{ [[}\DecValTok{1}\NormalTok{, }\DecValTok{2}\NormalTok{,}\DecValTok{0}\NormalTok{], [}\DecValTok{1}\NormalTok{,}\OperatorTok{{-}}\DecValTok{1}\NormalTok{,}\DecValTok{0}\NormalTok{]]}

\NormalTok{eliminacion\_gaussiana(Ab)}
\end{Highlighting}
\end{Shaded}

\begin{verbatim}
The autoreload extension is already loaded. To reload it, use:
  %reload_ext autoreload
[01-06 16:29:47][INFO] 
[[ 1  2  0]
 [ 0 -3  0]]
\end{verbatim}

\begin{verbatim}
array([ 0., -0.])
\end{verbatim}

\subsubsection{Literal b)}\label{literal-b}

\[x_1 + 2x_2 = 3\]

\[-2x_1 - 4x_2 = 6\]

\begin{Shaded}
\begin{Highlighting}[]

\NormalTok{A }\OperatorTok{=}\NormalTok{ [}
\NormalTok{    [}\DecValTok{1}\NormalTok{, }\DecValTok{2}\NormalTok{],}
\NormalTok{    [}\OperatorTok{{-}}\DecValTok{2}\NormalTok{, }\OperatorTok{{-}}\DecValTok{4}\NormalTok{]}
\NormalTok{]}

\NormalTok{b }\OperatorTok{=}\NormalTok{ [}\DecValTok{3}\NormalTok{, }\DecValTok{6}\NormalTok{]}

\NormalTok{x }\OperatorTok{=}\NormalTok{ np.linalg.solve(A, b)}
\BuiltInTok{print}\NormalTok{(x)}
\end{Highlighting}
\end{Shaded}

\begin{verbatim}
LinAlgError: Singular matrix
---------------------------------------------------------------------------
LinAlgError                               Traceback (most recent call last)
Cell In[4], line 8
      1 A = [
      2     [1, 2],
      3     [-2, -4]
      4 ]
      6 b = [3, 6]
----> 8 x = np.linalg.solve(A, b)
      9 print(x)

File c:\Users\acer\AppData\Local\Programs\Python\Python311\Lib\site-packages\numpy\linalg\_linalg.py:410, in solve(a, b)
    407 signature = 'DD->D' if isComplexType(t) else 'dd->d'
    408 with errstate(call=_raise_linalgerror_singular, invalid='call',
    409               over='ignore', divide='ignore', under='ignore'):
--> 410     r = gufunc(a, b, signature=signature)
    412 return wrap(r.astype(result_t, copy=False))

File c:\Users\acer\AppData\Local\Programs\Python\Python311\Lib\site-packages\numpy\linalg\_linalg.py:104, in _raise_linalgerror_singular(err, flag)
    103 def _raise_linalgerror_singular(err, flag):
--> 104     raise LinAlgError("Singular matrix")

LinAlgError: Singular matrix
\end{verbatim}

Sin solución.El error LinAlgError: Singular matrix indica que la matriz
𝐴 A no tiene inversa, lo que significa que no es posible resolver el
sistema de ecuaciones lineales utilizando np.linalg.solve.

\begin{Shaded}
\begin{Highlighting}[]
\OperatorTok{\%}\NormalTok{load\_ext autoreload}
\OperatorTok{\%}\NormalTok{autoreload }\DecValTok{2}
\ImportTok{from}\NormalTok{ src }\ImportTok{import}\NormalTok{ eliminacion\_gaussiana}

\NormalTok{Ab }\OperatorTok{=}\NormalTok{ [[}\DecValTok{1}\NormalTok{, }\DecValTok{2}\NormalTok{,}\DecValTok{3}\NormalTok{], [}\OperatorTok{{-}}\DecValTok{2}\NormalTok{,}\OperatorTok{{-}}\DecValTok{4}\NormalTok{,}\DecValTok{6}\NormalTok{]]}

\NormalTok{eliminacion\_gaussiana(Ab)}
\end{Highlighting}
\end{Shaded}

\begin{verbatim}
The autoreload extension is already loaded. To reload it, use:
  %reload_ext autoreload
[01-06 16:34:33][INFO] 
[[ 1  2  3]
 [ 0  0 12]]
\end{verbatim}

\begin{verbatim}
ValueError: No existe solución única.
---------------------------------------------------------------------------
ValueError                                Traceback (most recent call last)
Cell In[11], line 7
      3 from src import eliminacion_gaussiana
      5 Ab = [[1, 2,3], [-2,-4,6]]
----> 7 eliminacion_gaussiana(Ab)

File c:\Users\acer\OneDrive - Escuela Politécnica Nacional\4toSemestre\Metodos\Deberes\IIB\ELIMINACIONGAUS\src\linear_sist_methods.py:82, in eliminacion_gaussiana(A)
     79     logging.info(f"\n{A}")
     81 if A[n - 1, n - 1] == 0:
---> 82     raise ValueError("No existe solución única.")
     84     print(f"\n{A}")
     85 # --- Sustitución hacia atrás

ValueError: No existe solución única.
\end{verbatim}

\subsubsection{Literal c)}\label{literal-c}

\[2x_1 + x_2 = -1\]

\[x_1 + x_2 = 2\]

\[x_1 - 3x_2 = 5\]

\begin{Shaded}
\begin{Highlighting}[]

\ImportTok{import}\NormalTok{ numpy }\ImportTok{as}\NormalTok{ np}

\NormalTok{A }\OperatorTok{=}\NormalTok{ np.array([}
\NormalTok{    [}\DecValTok{2}\NormalTok{, }\DecValTok{1}\NormalTok{],}
\NormalTok{    [}\DecValTok{1}\NormalTok{, }\DecValTok{1}\NormalTok{],}
\NormalTok{    [}\DecValTok{1}\NormalTok{, }\OperatorTok{{-}}\DecValTok{3}\NormalTok{]}
\NormalTok{])}
\NormalTok{b }\OperatorTok{=}\NormalTok{ np.array([}\OperatorTok{{-}}\DecValTok{1}\NormalTok{, }\DecValTok{2}\NormalTok{, }\DecValTok{5}\NormalTok{])}

\NormalTok{x, residuals, rank, s }\OperatorTok{=}\NormalTok{ np.linalg.lstsq(A, b, rcond}\OperatorTok{=}\VariableTok{None}\NormalTok{)}
\BuiltInTok{print}\NormalTok{(x)}
\end{Highlighting}
\end{Shaded}

\begin{verbatim}
[ 0.83333333 -1.27272727]
\end{verbatim}

\begin{Shaded}
\begin{Highlighting}[]
\OperatorTok{\%}\NormalTok{load\_ext autoreload}
\OperatorTok{\%}\NormalTok{autoreload }\DecValTok{2}
\ImportTok{from}\NormalTok{ src }\ImportTok{import}\NormalTok{ eliminacion\_gaussiana}

\NormalTok{Ab }\OperatorTok{=}\NormalTok{ [[}\DecValTok{2}\NormalTok{, }\DecValTok{1}\NormalTok{,}\OperatorTok{{-}}\DecValTok{1}\NormalTok{], [}\DecValTok{1}\NormalTok{,}\DecValTok{1}\NormalTok{,}\DecValTok{2}\NormalTok{],[}\DecValTok{1}\NormalTok{,}\OperatorTok{{-}}\DecValTok{3}\NormalTok{,}\DecValTok{5}\NormalTok{]]}

\NormalTok{eliminacion\_gaussiana(Ab)}
\end{Highlighting}
\end{Shaded}

\begin{verbatim}
The autoreload extension is already loaded. To reload it, use:
  %reload_ext autoreload
\end{verbatim}

\begin{verbatim}
AssertionError: La matriz A debe ser de tamaño n-by-(n+1).
---------------------------------------------------------------------------
AssertionError                            Traceback (most recent call last)
Cell In[12], line 7
      3 from src import eliminacion_gaussiana
      5 Ab = [[2, 1,-1], [1,1,2],[1,-3,5]]
----> 7 eliminacion_gaussiana(Ab)

File c:\Users\acer\OneDrive - Escuela Politécnica Nacional\4toSemestre\Metodos\Deberes\IIB\ELIMINACIONGAUS\src\linear_sist_methods.py:43, in eliminacion_gaussiana(A)
     41     logging.debug("Convirtiendo A a numpy array.")
     42     A = np.array(A)
---> 43 assert A.shape[0] == A.shape[1] - 1, "La matriz A debe ser de tamaño n-by-(n+1)."
     44 n = A.shape[0]
     46 for i in range(0, n - 1):  # loop por columna
     47 
     48     # --- encontrar pivote

AssertionError: La matriz A debe ser de tamaño n-by-(n+1).
\end{verbatim}

Para este caso, existen soluciones infinitas

\subsubsection{Literal d)}\label{literal-d}

\[2x_1 + x_2 + x_3 = 1\]

\[2x_1 + 4x_2 - x_3= -1\]

\begin{Shaded}
\begin{Highlighting}[]

\NormalTok{A }\OperatorTok{=}\NormalTok{ [}
\NormalTok{    [}\DecValTok{2}\NormalTok{, }\DecValTok{1}\NormalTok{, }\DecValTok{1}\NormalTok{],}
\NormalTok{    [}\DecValTok{2}\NormalTok{, }\DecValTok{4}\NormalTok{, }\OperatorTok{{-}}\DecValTok{1}\NormalTok{]}
\NormalTok{]}

\NormalTok{b }\OperatorTok{=}\NormalTok{ [}\DecValTok{1}\NormalTok{, }\OperatorTok{{-}}\DecValTok{1}\NormalTok{]}

\NormalTok{x }\OperatorTok{=}\NormalTok{ np.linalg.solve(A, b)}
\BuiltInTok{print}\NormalTok{(x)}
\end{Highlighting}
\end{Shaded}

\begin{verbatim}
LinAlgError: Last 2 dimensions of the array must be square
---------------------------------------------------------------------------
LinAlgError                               Traceback (most recent call last)
Cell In[23], line 8
      1 A = [
      2     [2, 1, 1],
      3     [2, 4, -1]
      4 ]
      6 b = [1, -1]
----> 8 x = np.linalg.solve(A, b)
      9 print(x)

File ~\anaconda3\Lib\site-packages\numpy\linalg\linalg.py:396, in solve(a, b)
    394 a, _ = _makearray(a)
    395 _assert_stacked_2d(a)
--> 396 _assert_stacked_square(a)
    397 b, wrap = _makearray(b)
    398 t, result_t = _commonType(a, b)

File ~\anaconda3\Lib\site-packages\numpy\linalg\linalg.py:213, in _assert_stacked_square(*arrays)
    211 m, n = a.shape[-2:]
    212 if m != n:
--> 213     raise LinAlgError('Last 2 dimensions of the array must be square')

LinAlgError: Last 2 dimensions of the array must be square
\end{verbatim}

Soluciones infinitas

\subsection{Ejercicio 2}\label{ejercicio-2}

Utilice la eliminación gaussiana con sustitución hacia atrás y
aritmética de redondeo de dos dígitos para resolver los siguientes
sistemas lineales. No reordene las ecuaciones. (La solución exacta para
cada sistema es \(x_1 = -1\), \(x_2 = 2\), \(x_3 = 3\).)

\subsubsection{Literal a)}\label{literal-a-1}

\[-x_1 + 4x_2 + x_3 = 8\]

\[\frac{5}{3}x_1 + \frac{2}{3}x_2 - \frac{2}{3}x_3= 1\]

\[2x_1 + x_2 + 4x_3 = 11\]

\begin{Shaded}
\begin{Highlighting}[]

\NormalTok{A }\OperatorTok{=}\NormalTok{ [}
\NormalTok{    [}\OperatorTok{{-}}\DecValTok{1}\NormalTok{, }\DecValTok{4}\NormalTok{, }\DecValTok{1}\NormalTok{],}
\NormalTok{    [}\BuiltInTok{round}\NormalTok{(}\DecValTok{5}\OperatorTok{/}\DecValTok{3}\NormalTok{, }\DecValTok{2}\NormalTok{), }\BuiltInTok{round}\NormalTok{(}\DecValTok{2}\OperatorTok{/}\DecValTok{3}\NormalTok{, }\DecValTok{2}\NormalTok{), }\BuiltInTok{round}\NormalTok{(}\DecValTok{2}\OperatorTok{/}\DecValTok{3}\NormalTok{, }\DecValTok{2}\NormalTok{)],}
\NormalTok{    [}\DecValTok{2}\NormalTok{, }\DecValTok{1}\NormalTok{, }\DecValTok{4}\NormalTok{]}
\NormalTok{]}

\NormalTok{b }\OperatorTok{=}\NormalTok{ [}\DecValTok{8}\NormalTok{, }\DecValTok{1}\NormalTok{, }\DecValTok{11}\NormalTok{]}

\NormalTok{x }\OperatorTok{=}\NormalTok{ np.linalg.solve(A, b)}
\BuiltInTok{print}\NormalTok{(x)}
\end{Highlighting}
\end{Shaded}

\begin{verbatim}
[-1.00651042  0.99739583  3.00390625]
\end{verbatim}

\subsubsection{Literal b)}\label{literal-b-1}

\[4x_1 + 2x_2 - x_3 = -5\]

\[\frac{1}{9}x_1 + \frac{1}{9}x_2 - \frac{1}{3}x_3= -1\]

\[x_1 + 4x_2 + 2x_3 = 9\]

\begin{Shaded}
\begin{Highlighting}[]

\NormalTok{A }\OperatorTok{=}\NormalTok{ [}
\NormalTok{    [}\DecValTok{4}\NormalTok{, }\DecValTok{2}\NormalTok{, }\OperatorTok{{-}}\DecValTok{1}\NormalTok{],}
\NormalTok{    [}\BuiltInTok{round}\NormalTok{(}\DecValTok{1}\OperatorTok{/}\DecValTok{9}\NormalTok{, }\DecValTok{2}\NormalTok{), }\BuiltInTok{round}\NormalTok{(}\DecValTok{1}\OperatorTok{/}\DecValTok{9}\NormalTok{, }\DecValTok{2}\NormalTok{), }\BuiltInTok{round}\NormalTok{(}\DecValTok{1}\OperatorTok{/}\DecValTok{3}\NormalTok{, }\DecValTok{2}\NormalTok{)],}
\NormalTok{    [}\DecValTok{1}\NormalTok{, }\DecValTok{4}\NormalTok{, }\DecValTok{2}\NormalTok{]}
\NormalTok{]}

\NormalTok{b }\OperatorTok{=}\NormalTok{ [}\OperatorTok{{-}}\DecValTok{5}\NormalTok{, }\OperatorTok{{-}}\DecValTok{1}\NormalTok{, }\DecValTok{9}\NormalTok{]}

\NormalTok{x }\OperatorTok{=}\NormalTok{ np.linalg.solve(A, b)}
\BuiltInTok{print}\NormalTok{(x)}
\end{Highlighting}
\end{Shaded}

\begin{verbatim}
[-4.52993348  4.97117517 -3.17738359]
\end{verbatim}

\subsection{Ejercicio 3}\label{ejercicio-3}

Utilice el algoritmo de eliminación gaussiana para resolver, de ser
posible, los siguientes sistemas de ecuaciones lineales y muestre los
intercambios de fila necesarios.

\begin{Shaded}
\begin{Highlighting}[]
\KeywordTok{def}\NormalTok{ eliminacion\_gaussiana(A: np.ndarray) }\OperatorTok{{-}\textgreater{}}\NormalTok{ np.ndarray:}
\NormalTok{    A }\OperatorTok{=}\NormalTok{ np.array(A)}
    \ControlFlowTok{assert}\NormalTok{ A.shape[}\DecValTok{0}\NormalTok{] }\OperatorTok{==}\NormalTok{ A.shape[}\DecValTok{1}\NormalTok{] }\OperatorTok{{-}} \DecValTok{1}\NormalTok{, }\StringTok{"La matriz A debe ser de tamaño n{-}by{-}(n+1)."}
\NormalTok{    n }\OperatorTok{=}\NormalTok{ A.shape[}\DecValTok{0}\NormalTok{]}

    \ControlFlowTok{for}\NormalTok{ i }\KeywordTok{in} \BuiltInTok{range}\NormalTok{(}\DecValTok{0}\NormalTok{, n }\OperatorTok{{-}} \DecValTok{1}\NormalTok{):  }\CommentTok{\# loop por columna}

        \CommentTok{\# {-}{-}{-} encontrar pivote}
\NormalTok{        p }\OperatorTok{=} \VariableTok{None}  \CommentTok{\# default, first element}
        \ControlFlowTok{for}\NormalTok{ pi }\KeywordTok{in} \BuiltInTok{range}\NormalTok{(i, n):}
            \ControlFlowTok{if}\NormalTok{ A[pi, i] }\OperatorTok{==} \DecValTok{0}\NormalTok{:}
                \CommentTok{\# must be nonzero}
                \ControlFlowTok{continue}

            \ControlFlowTok{if}\NormalTok{ p }\KeywordTok{is} \VariableTok{None}\NormalTok{:}
                \CommentTok{\# first nonzero element}
\NormalTok{                p }\OperatorTok{=}\NormalTok{ pi}
                \ControlFlowTok{continue}

            \ControlFlowTok{if} \BuiltInTok{abs}\NormalTok{(A[pi, i]) }\OperatorTok{\textless{}} \BuiltInTok{abs}\NormalTok{(A[p, i]):}
\NormalTok{                p }\OperatorTok{=}\NormalTok{ pi}

        \ControlFlowTok{if}\NormalTok{ p }\KeywordTok{is} \VariableTok{None}\NormalTok{:}
            \CommentTok{\# no pivot found.}
            \ControlFlowTok{raise} \PreprocessorTok{ValueError}\NormalTok{(}\StringTok{"No existe solución única."}\NormalTok{)}

        \ControlFlowTok{if}\NormalTok{ p }\OperatorTok{!=}\NormalTok{ i:}
            \CommentTok{\# swap rows}
            \BuiltInTok{print}\NormalTok{(}\SpecialStringTok{f"Intercambiando filas }\SpecialCharTok{\{}\NormalTok{i}\SpecialCharTok{\}}\SpecialStringTok{ y }\SpecialCharTok{\{}\NormalTok{p}\SpecialCharTok{\}}\SpecialStringTok{"}\NormalTok{)}
\NormalTok{            \_aux }\OperatorTok{=}\NormalTok{ A[i, :].copy()}
\NormalTok{            A[i, :] }\OperatorTok{=}\NormalTok{ A[p, :].copy()}
\NormalTok{            A[p, :] }\OperatorTok{=}\NormalTok{ \_aux}

        \CommentTok{\# {-}{-}{-} Eliminación: loop por fila}
        \ControlFlowTok{for}\NormalTok{ j }\KeywordTok{in} \BuiltInTok{range}\NormalTok{(i }\OperatorTok{+} \DecValTok{1}\NormalTok{, n):}
\NormalTok{            m }\OperatorTok{=}\NormalTok{ A[j, i] }\OperatorTok{/}\NormalTok{ A[i, i]}
\NormalTok{            A[j, i:] }\OperatorTok{=}\NormalTok{ A[j, i:] }\OperatorTok{{-}}\NormalTok{ m }\OperatorTok{*}\NormalTok{ A[i, i:]}


    \ControlFlowTok{if}\NormalTok{ A[n }\OperatorTok{{-}} \DecValTok{1}\NormalTok{, n }\OperatorTok{{-}} \DecValTok{1}\NormalTok{] }\OperatorTok{==} \DecValTok{0}\NormalTok{:}
        \ControlFlowTok{raise} \PreprocessorTok{ValueError}\NormalTok{(}\StringTok{"No existe solución única."}\NormalTok{)}

        \BuiltInTok{print}\NormalTok{(}\SpecialStringTok{f"}\CharTok{\textbackslash{}n}\SpecialCharTok{\{}\NormalTok{A}\SpecialCharTok{\}}\SpecialStringTok{"}\NormalTok{)}
    \CommentTok{\# {-}{-}{-} Sustitución hacia atrás}
\NormalTok{    solucion }\OperatorTok{=}\NormalTok{ np.zeros(n)}
\NormalTok{    solucion[n }\OperatorTok{{-}} \DecValTok{1}\NormalTok{] }\OperatorTok{=}\NormalTok{ A[n }\OperatorTok{{-}} \DecValTok{1}\NormalTok{, n] }\OperatorTok{/}\NormalTok{ A[n }\OperatorTok{{-}} \DecValTok{1}\NormalTok{, n }\OperatorTok{{-}} \DecValTok{1}\NormalTok{]}

    \ControlFlowTok{for}\NormalTok{ i }\KeywordTok{in} \BuiltInTok{range}\NormalTok{(n }\OperatorTok{{-}} \DecValTok{2}\NormalTok{, }\OperatorTok{{-}}\DecValTok{1}\NormalTok{, }\OperatorTok{{-}}\DecValTok{1}\NormalTok{):}
\NormalTok{        suma }\OperatorTok{=} \DecValTok{0}
        \ControlFlowTok{for}\NormalTok{ j }\KeywordTok{in} \BuiltInTok{range}\NormalTok{(i }\OperatorTok{+} \DecValTok{1}\NormalTok{, n):}
\NormalTok{            suma }\OperatorTok{+=}\NormalTok{ A[i, j] }\OperatorTok{*}\NormalTok{ solucion[j]}
\NormalTok{        solucion[i] }\OperatorTok{=}\NormalTok{ (A[i, n] }\OperatorTok{{-}}\NormalTok{ suma) }\OperatorTok{/}\NormalTok{ A[i, i]}

    \ControlFlowTok{return}\NormalTok{ solucion}
\end{Highlighting}
\end{Shaded}

\subsubsection{Literal a)}\label{literal-a-2}

\[x_1 - x_2 + 3x_3 = 2\]

\[3x_1 - 3x_2 + 1x_3= - 1\]

\[x_1 + x_2  = 3\]

\begin{Shaded}
\begin{Highlighting}[]

\NormalTok{A }\OperatorTok{=}\NormalTok{ [}
\NormalTok{    [}\DecValTok{1}\NormalTok{, }\OperatorTok{{-}}\DecValTok{1}\NormalTok{, }\DecValTok{3}\NormalTok{, }\DecValTok{2}\NormalTok{],}
\NormalTok{    [}\DecValTok{3}\NormalTok{, }\OperatorTok{{-}}\DecValTok{3}\NormalTok{, }\DecValTok{1}\NormalTok{, }\OperatorTok{{-}}\DecValTok{1}\NormalTok{],}
\NormalTok{    [}\DecValTok{1}\NormalTok{, }\DecValTok{1}\NormalTok{, }\DecValTok{0}\NormalTok{, }\DecValTok{3}\NormalTok{]}
\NormalTok{]}

\NormalTok{x }\OperatorTok{=}\NormalTok{ eliminacion\_gaussiana(A)}
\BuiltInTok{print}\NormalTok{(x)}
\end{Highlighting}
\end{Shaded}

\begin{verbatim}
Intercambiando filas 1 y 2
[1.1875 1.8125 0.875 ]
\end{verbatim}

\subsubsection{Literal b)}\label{literal-b-2}

\[2x_1 - 1.5x_2 + 3x_3 = 1\]

\[-x_1 + 2x_3 = 3\]

\[4x_1 - 4.5x_2 + 5x_3 = 1\]

\begin{Shaded}
\begin{Highlighting}[]

\NormalTok{A }\OperatorTok{=}\NormalTok{ [}
\NormalTok{    [}\DecValTok{2}\NormalTok{, }\OperatorTok{{-}}\FloatTok{1.5}\NormalTok{, }\DecValTok{3}\NormalTok{, }\DecValTok{1}\NormalTok{],}
\NormalTok{    [}\OperatorTok{{-}}\DecValTok{1}\NormalTok{, }\DecValTok{0}\NormalTok{, }\DecValTok{2}\NormalTok{, }\DecValTok{3}\NormalTok{],}
\NormalTok{    [}\DecValTok{4}\NormalTok{, }\OperatorTok{{-}}\FloatTok{4.5}\NormalTok{, }\DecValTok{5}\NormalTok{, }\DecValTok{1}\NormalTok{]}
\NormalTok{]}

\NormalTok{x }\OperatorTok{=}\NormalTok{ eliminacion\_gaussiana(A)}
\BuiltInTok{print}\NormalTok{(x)}
\end{Highlighting}
\end{Shaded}

\begin{verbatim}
Intercambiando filas 0 y 1
[-1. -0.  1.]
\end{verbatim}

\subsubsection{Literal c)}\label{literal-c-1}

\[2x_1 = 3\]

\[x_1 + 1.5x_2= 4.5\]

\[- 3x_2 - 0.5x_3 = -6.6\]

\[2x_1 - 2x_2 + x_3 + x_4 = 0.8\]

\begin{Shaded}
\begin{Highlighting}[]

\NormalTok{A }\OperatorTok{=}\NormalTok{ [}
\NormalTok{    [}\DecValTok{2}\NormalTok{, }\DecValTok{0}\NormalTok{, }\DecValTok{0}\NormalTok{, }\DecValTok{0}\NormalTok{, }\DecValTok{3}\NormalTok{],}
\NormalTok{    [}\DecValTok{1}\NormalTok{, }\FloatTok{1.5}\NormalTok{, }\DecValTok{0}\NormalTok{, }\DecValTok{0}\NormalTok{, }\FloatTok{4.5}\NormalTok{],}
\NormalTok{    [}\DecValTok{0}\NormalTok{, }\OperatorTok{{-}}\DecValTok{3}\NormalTok{, }\FloatTok{0.5}\NormalTok{, }\DecValTok{0}\NormalTok{, }\OperatorTok{{-}}\FloatTok{6.6}\NormalTok{],}
\NormalTok{    [}\DecValTok{2}\NormalTok{, }\OperatorTok{{-}}\DecValTok{2}\NormalTok{, }\DecValTok{1}\NormalTok{, }\DecValTok{1}\NormalTok{, }\FloatTok{0.8}\NormalTok{]}
\NormalTok{]}

\NormalTok{x }\OperatorTok{=}\NormalTok{ eliminacion\_gaussiana(A)}
\BuiltInTok{print}\NormalTok{(x)}
\end{Highlighting}
\end{Shaded}

\begin{verbatim}
Intercambiando filas 0 y 1
[ 1.5  2.  -1.2  3. ]
\end{verbatim}

\subsubsection{Literal d)}\label{literal-d-1}

\[x_1 + x_2 + x_4 = 2\]

\[2x_1 + x_2 - x_3 + x_4 = 1\]

\[4x_1 - x_2 - 2x_3 + 2x_4 = 0\]

\[3x_1 - x_2 - x_3 + 2x_4 = -3\]

\begin{Shaded}
\begin{Highlighting}[]

\NormalTok{A }\OperatorTok{=}\NormalTok{ [}
\NormalTok{    [}\DecValTok{1}\NormalTok{, }\DecValTok{1}\NormalTok{, }\DecValTok{0}\NormalTok{, }\DecValTok{1}\NormalTok{, }\DecValTok{2}\NormalTok{],}
\NormalTok{    [}\DecValTok{2}\NormalTok{, }\DecValTok{1}\NormalTok{, }\OperatorTok{{-}}\DecValTok{1}\NormalTok{, }\DecValTok{1}\NormalTok{, }\DecValTok{1}\NormalTok{],}
\NormalTok{    [}\DecValTok{4}\NormalTok{, }\OperatorTok{{-}}\DecValTok{1}\NormalTok{, }\OperatorTok{{-}}\DecValTok{2}\NormalTok{, }\DecValTok{2}\NormalTok{, }\DecValTok{0}\NormalTok{],}
\NormalTok{    [}\DecValTok{3}\NormalTok{, }\OperatorTok{{-}}\DecValTok{1}\NormalTok{, }\OperatorTok{{-}}\DecValTok{1}\NormalTok{, }\DecValTok{2}\NormalTok{, }\OperatorTok{{-}}\DecValTok{3}\NormalTok{]}
\NormalTok{]}

\NormalTok{x }\OperatorTok{=}\NormalTok{ eliminacion\_gaussiana(A)}
\BuiltInTok{print}\NormalTok{(x)}
\end{Highlighting}
\end{Shaded}

\begin{verbatim}
ValueError: No existe solución única.
---------------------------------------------------------------------------
ValueError                                Traceback (most recent call last)
Cell In[128], line 8
      1 A = [
      2     [1, 1, 0, 1, 2],
      3     [2, 1, -1, 1, 1],
      4     [4, -1, -2, 2, 0],
      5     [3, -1, -1, 2, -3]
      6 ]
----> 8 x = eliminacion_gaussiana(A)
      9 print(x)

Cell In[112], line 41, in eliminacion_gaussiana(A)
     37         A[j, i:] = A[j, i:] - m * A[i, i:]
     40 if A[n - 1, n - 1] == 0:
---> 41     raise ValueError("No existe solución única.")
     43     print(f"\n{A}")
     44 # --- Sustitución hacia atrás

ValueError: No existe solución única.
\end{verbatim}

\subsection{Ejercicio 4}\label{ejercicio-4}

Use el algoritmo de eliminación gaussiana y la aritmética computacional
de precisión de 32 bits para resolver los siguientes sistemas lineales.

Para esto, utilizaremos la siguiente función:

\begin{Shaded}
\begin{Highlighting}[]
\KeywordTok{def}\NormalTok{ eliminacion\_gaussiana(A: np.ndarray) }\OperatorTok{{-}\textgreater{}}\NormalTok{ np.ndarray:}
\NormalTok{    A }\OperatorTok{=}\NormalTok{ np.array(A)}
    
    \ControlFlowTok{assert}\NormalTok{ A.shape[}\DecValTok{0}\NormalTok{] }\OperatorTok{==}\NormalTok{ A.shape[}\DecValTok{1}\NormalTok{] }\OperatorTok{{-}} \DecValTok{1}\NormalTok{, }\StringTok{"La matriz A debe ser de tamaño n{-}by{-}(n+1)."}
\NormalTok{    n }\OperatorTok{=}\NormalTok{ A.shape[}\DecValTok{0}\NormalTok{]}

    \ControlFlowTok{for}\NormalTok{ i }\KeywordTok{in} \BuiltInTok{range}\NormalTok{(}\DecValTok{0}\NormalTok{, n }\OperatorTok{{-}} \DecValTok{1}\NormalTok{):  }\CommentTok{\# loop por columna}

        \CommentTok{\# {-}{-}{-} encontrar pivote}
\NormalTok{        p }\OperatorTok{=} \VariableTok{None}  \CommentTok{\# default, first element}
        \ControlFlowTok{for}\NormalTok{ pi }\KeywordTok{in} \BuiltInTok{range}\NormalTok{(i, n):}
            \ControlFlowTok{if}\NormalTok{ A[pi, i] }\OperatorTok{==} \DecValTok{0}\NormalTok{:}
                \CommentTok{\# must be nonzero}
                \ControlFlowTok{continue}

            \ControlFlowTok{if}\NormalTok{ p }\KeywordTok{is} \VariableTok{None}\NormalTok{:}
                \CommentTok{\# first nonzero element}
\NormalTok{                p }\OperatorTok{=}\NormalTok{ pi}
                \ControlFlowTok{continue}

            \ControlFlowTok{if} \BuiltInTok{abs}\NormalTok{(A[pi, i]) }\OperatorTok{\textless{}} \BuiltInTok{abs}\NormalTok{(A[p, i]):}
\NormalTok{                p }\OperatorTok{=}\NormalTok{ pi}

        \ControlFlowTok{if}\NormalTok{ p }\KeywordTok{is} \VariableTok{None}\NormalTok{:}
            \CommentTok{\# no pivot found.}
            \ControlFlowTok{raise} \PreprocessorTok{ValueError}\NormalTok{(}\StringTok{"No existe solución única."}\NormalTok{)}

        \ControlFlowTok{if}\NormalTok{ p }\OperatorTok{!=}\NormalTok{ i:}
            \CommentTok{\# swap rows}
            \BuiltInTok{print}\NormalTok{(}\SpecialStringTok{f"Intercambiando filas }\SpecialCharTok{\{}\NormalTok{i}\SpecialCharTok{\}}\SpecialStringTok{ y }\SpecialCharTok{\{}\NormalTok{p}\SpecialCharTok{\}}\SpecialStringTok{"}\NormalTok{)}
\NormalTok{            \_aux }\OperatorTok{=}\NormalTok{ A[i, :].copy()}
\NormalTok{            A[i, :] }\OperatorTok{=}\NormalTok{ A[p, :].copy()}
\NormalTok{            A[p, :] }\OperatorTok{=}\NormalTok{ \_aux}

        \CommentTok{\# {-}{-}{-} Eliminación: loop por fila}
        \ControlFlowTok{for}\NormalTok{ j }\KeywordTok{in} \BuiltInTok{range}\NormalTok{(i }\OperatorTok{+} \DecValTok{1}\NormalTok{, n):}
\NormalTok{            m }\OperatorTok{=}\NormalTok{ A[j, i] }\OperatorTok{/}\NormalTok{ A[i, i]}
\NormalTok{            A[j, i:] }\OperatorTok{=}\NormalTok{ A[j, i:] }\OperatorTok{{-}}\NormalTok{ m }\OperatorTok{*}\NormalTok{ A[i, i:]}

    \ControlFlowTok{if}\NormalTok{ A[n }\OperatorTok{{-}} \DecValTok{1}\NormalTok{, n }\OperatorTok{{-}} \DecValTok{1}\NormalTok{] }\OperatorTok{==} \DecValTok{0}\NormalTok{:}
        \ControlFlowTok{raise} \PreprocessorTok{ValueError}\NormalTok{(}\StringTok{"No existe solución única."}\NormalTok{)}

    \CommentTok{\# {-}{-}{-} Sustitución hacia atrás}
\NormalTok{    solucion }\OperatorTok{=}\NormalTok{ np.zeros(n, dtype}\OperatorTok{=}\NormalTok{np.float32)}
\NormalTok{    solucion[n }\OperatorTok{{-}} \DecValTok{1}\NormalTok{] }\OperatorTok{=}\NormalTok{ A[n }\OperatorTok{{-}} \DecValTok{1}\NormalTok{, n] }\OperatorTok{/}\NormalTok{ A[n }\OperatorTok{{-}} \DecValTok{1}\NormalTok{, n }\OperatorTok{{-}} \DecValTok{1}\NormalTok{]}

    \ControlFlowTok{for}\NormalTok{ i }\KeywordTok{in} \BuiltInTok{range}\NormalTok{(n }\OperatorTok{{-}} \DecValTok{2}\NormalTok{, }\OperatorTok{{-}}\DecValTok{1}\NormalTok{, }\OperatorTok{{-}}\DecValTok{1}\NormalTok{):}
\NormalTok{        suma }\OperatorTok{=} \DecValTok{0}
        \ControlFlowTok{for}\NormalTok{ j }\KeywordTok{in} \BuiltInTok{range}\NormalTok{(i }\OperatorTok{+} \DecValTok{1}\NormalTok{, n):}
\NormalTok{            suma }\OperatorTok{+=}\NormalTok{ A[i, j] }\OperatorTok{*}\NormalTok{ solucion[j]}
\NormalTok{        solucion[i] }\OperatorTok{=}\NormalTok{ (A[i, n] }\OperatorTok{{-}}\NormalTok{ suma) }\OperatorTok{/}\NormalTok{ A[i, i]}

    \ControlFlowTok{return}\NormalTok{ solucion}
\end{Highlighting}
\end{Shaded}

\subsubsection{Literal a)}\label{literal-a-3}

\[\frac{1}{4}x_1 + \frac{1}{5}x_2 + \frac{1}{6}x_3 = 9\]

\[\frac{1}{3}x_1 + \frac{1}{4}x_2 + \frac{1}{5}x_3= 8\]

\[\frac{1}{2}x_1 + x_2 + 2x_3 = 8\]

\begin{Shaded}
\begin{Highlighting}[]

\NormalTok{A }\OperatorTok{=}\NormalTok{ np.array([[}\DecValTok{1}\OperatorTok{/}\DecValTok{4}\NormalTok{, }\DecValTok{1}\OperatorTok{/}\DecValTok{5}\NormalTok{, }\DecValTok{1}\OperatorTok{/}\DecValTok{6}\NormalTok{, }\DecValTok{9}\NormalTok{],}
\NormalTok{              [}\DecValTok{1}\OperatorTok{/}\DecValTok{3}\NormalTok{, }\DecValTok{1}\OperatorTok{/}\DecValTok{4}\NormalTok{, }\DecValTok{1}\OperatorTok{/}\DecValTok{5}\NormalTok{, }\DecValTok{8}\NormalTok{],}
\NormalTok{              [}\DecValTok{1}\OperatorTok{/}\DecValTok{2}\NormalTok{, }\DecValTok{1}\NormalTok{, }\DecValTok{2}\NormalTok{, }\DecValTok{8}\NormalTok{]], dtype}\OperatorTok{=}\NormalTok{np.float32)}

\NormalTok{solucion }\OperatorTok{=}\NormalTok{ eliminacion\_gaussiana(A)}
\BuiltInTok{print}\NormalTok{(solucion)}
\end{Highlighting}
\end{Shaded}

\begin{verbatim}
[-227.07666  476.92264 -177.69217]
\end{verbatim}

\subsubsection{Literal b)}\label{literal-b-3}

\[3.333x_1 + 15920x_2 - 10.333x_3 = 15913\]

\[2.222x_1 + 16.71x_2 + 9.612x_3 = 28.544\]

\[1.5611x_1 + 5.1791x_2 + 1.6852x_3 = 8.4254\]

\begin{Shaded}
\begin{Highlighting}[]

\NormalTok{A }\OperatorTok{=}\NormalTok{ np.array([[}\FloatTok{3.333}\NormalTok{, }\DecValTok{15920}\NormalTok{, }\FloatTok{10.333}\NormalTok{, }\DecValTok{15913}\NormalTok{],}
\NormalTok{              [}\FloatTok{2.222}\NormalTok{, }\FloatTok{16.71}\NormalTok{, }\FloatTok{9.612}\NormalTok{, }\FloatTok{28.544}\NormalTok{],}
\NormalTok{              [}\FloatTok{1.5611}\NormalTok{, }\FloatTok{5.1791}\NormalTok{, }\FloatTok{1.6852}\NormalTok{, }\FloatTok{8.4254}\NormalTok{]], dtype}\OperatorTok{=}\NormalTok{np.float32)}

\NormalTok{solucion }\OperatorTok{=}\NormalTok{ eliminacion\_gaussiana(A)}
\BuiltInTok{print}\NormalTok{(solucion)}
\end{Highlighting}
\end{Shaded}

\begin{verbatim}
Intercambiando filas 0 y 2
[1.0024954 0.9987004 1.0016824]
\end{verbatim}

\subsubsection{Literal c)}\label{literal-c-2}

\[x_1 + \frac{1}{2}x_2 + \frac{1}{3}x_3 + \frac{1}{4}x_4 = \frac{1}{6}\]

\[\frac{1}{2}x_1 + \frac{1}{3}x_2 + \frac{1}{4}x_3 + \frac{1}{5}x_4 = \frac{1}{7}\]

\[\frac{1}{3}x_1 + \frac{1}{4}x_2 + \frac{1}{5}x_3 + \frac{1}{6}x_4 = \frac{1}{8}\]

\[\frac{1}{4}x_1 + \frac{1}{5}x_2 + \frac{1}{6}x_3 + \frac{1}{7}x_4 = \frac{1}{9}\]

\begin{Shaded}
\begin{Highlighting}[]

\NormalTok{A }\OperatorTok{=}\NormalTok{ np.array([[}\DecValTok{1}\NormalTok{, }\DecValTok{1}\OperatorTok{/}\DecValTok{2}\NormalTok{, }\DecValTok{1}\OperatorTok{/}\DecValTok{3}\NormalTok{, }\DecValTok{1}\OperatorTok{/}\DecValTok{4}\NormalTok{, }\DecValTok{1}\OperatorTok{/}\DecValTok{6}\NormalTok{],}
\NormalTok{              [}\DecValTok{1}\OperatorTok{/}\DecValTok{2}\NormalTok{, }\DecValTok{1}\OperatorTok{/}\DecValTok{3}\NormalTok{, }\DecValTok{1}\OperatorTok{/}\DecValTok{4}\NormalTok{, }\DecValTok{1}\OperatorTok{/}\DecValTok{5}\NormalTok{, }\DecValTok{1}\OperatorTok{/}\DecValTok{7}\NormalTok{],}
\NormalTok{              [}\DecValTok{1}\OperatorTok{/}\DecValTok{3}\NormalTok{, }\DecValTok{1}\OperatorTok{/}\DecValTok{4}\NormalTok{, }\DecValTok{1}\OperatorTok{/}\DecValTok{5}\NormalTok{, }\DecValTok{1}\OperatorTok{/}\DecValTok{6}\NormalTok{, }\DecValTok{1}\OperatorTok{/}\DecValTok{8}\NormalTok{],}
\NormalTok{              [}\DecValTok{1}\OperatorTok{/}\DecValTok{4}\NormalTok{, }\DecValTok{1}\OperatorTok{/}\DecValTok{5}\NormalTok{, }\DecValTok{1}\OperatorTok{/}\DecValTok{6}\NormalTok{, }\DecValTok{1}\OperatorTok{/}\DecValTok{7}\NormalTok{, }\DecValTok{1}\OperatorTok{/}\DecValTok{9}\NormalTok{]], dtype}\OperatorTok{=}\NormalTok{np.float32)}

\NormalTok{solucion }\OperatorTok{=}\NormalTok{ eliminacion\_gaussiana(A)}
\BuiltInTok{print}\NormalTok{(solucion)}
\end{Highlighting}
\end{Shaded}

\begin{verbatim}
Intercambiando filas 0 y 3
Intercambiando filas 1 y 2
[-0.03174073  0.5951853  -2.3808312   2.7777011 ]
\end{verbatim}

\subsubsection{Literal d)}\label{literal-d-2}

\[2x_1 + x_2 - x_3 + x_4 - 3x_5 = 7\]

\[x_1 + 2x_3 - x_4 + x_5 = 2\]

\[- 2x_2 - x_3 + x_4 - x_5 = -5\]

\[3x_1 + x_2 - 4x_3 + 5x_5 = 6\]

\[x_1 - x_2 - x_3 - x_4 + x_5= -3\]

\begin{Shaded}
\begin{Highlighting}[]

\NormalTok{A }\OperatorTok{=}\NormalTok{ np.array([[}\DecValTok{2}\NormalTok{, }\DecValTok{1}\NormalTok{, }\OperatorTok{{-}}\DecValTok{1}\NormalTok{, }\DecValTok{1}\NormalTok{, }\OperatorTok{{-}}\DecValTok{3}\NormalTok{, }\DecValTok{7}\NormalTok{],}
\NormalTok{              [}\DecValTok{1}\NormalTok{, }\DecValTok{0}\NormalTok{, }\DecValTok{2}\NormalTok{, }\OperatorTok{{-}}\DecValTok{1}\NormalTok{, }\DecValTok{1}\NormalTok{, }\DecValTok{2}\NormalTok{],}
\NormalTok{              [}\DecValTok{0}\NormalTok{, }\OperatorTok{{-}}\DecValTok{2}\NormalTok{, }\OperatorTok{{-}}\DecValTok{1}\NormalTok{, }\DecValTok{1}\NormalTok{, }\OperatorTok{{-}}\DecValTok{1}\NormalTok{, }\OperatorTok{{-}}\DecValTok{5}\NormalTok{],}
\NormalTok{              [}\DecValTok{3}\NormalTok{, }\DecValTok{1}\NormalTok{, }\OperatorTok{{-}}\DecValTok{4}\NormalTok{, }\DecValTok{0}\NormalTok{, }\DecValTok{5}\NormalTok{, }\DecValTok{6}\NormalTok{],}
\NormalTok{              [}\DecValTok{1}\NormalTok{, }\OperatorTok{{-}}\DecValTok{1}\NormalTok{, }\OperatorTok{{-}}\DecValTok{1}\NormalTok{, }\OperatorTok{{-}}\DecValTok{1}\NormalTok{, }\DecValTok{1}\NormalTok{, }\OperatorTok{{-}}\DecValTok{3}\NormalTok{]], dtype}\OperatorTok{=}\NormalTok{np.float32)}

\NormalTok{solucion }\OperatorTok{=}\NormalTok{ eliminacion\_gaussiana(A)}
\BuiltInTok{print}\NormalTok{(solucion)}
\end{Highlighting}
\end{Shaded}

\begin{verbatim}
Intercambiando filas 0 y 1
Intercambiando filas 2 y 3
Intercambiando filas 3 y 4
[1.883041   2.8070176  0.73099416 1.4385966  0.09356727]
\end{verbatim}

\subsection{Ejercicio 5}\label{ejercicio-5}

Dado el sistema lineal:

\[x_1 - x_2 + \alpha x_3 = -2\]

\[- x_1 + 2x_2 - \alpha x_3 = 3\]

\[\alpha x_1 + x_2 + x_3 = 2\]

\subsubsection{Literal a)}\label{literal-a-4}

Encuentre el valor(es) de \(\alpha\) para los que el sistema no tiene
soluciones.

Se puede obtener \(\alpha\) si calculamos el determinante de esta matriz
y la igualamos a cero, entonces:

\[det(A) = (2 + \alpha) + (-1 + \alpha) + \alpha (- 1 - 2 \alpha) = 0\]

\[det(A) = 2 \alpha^{2} - \alpha + 1 = 0\]

\[\alpha = 1      ;     \alpha = - \frac{1}{2}\]

Si \(\alpha\) tiene alguno de los valores del conjunto:
\(\set{-\frac{1}{2}, 1}\), entonces el sistema no tiene solución.

\subsubsection{Literal b)}\label{literal-b-4}

Encuentre el valor(es) de 𝛼 para los que el sistema tiene un número
infinito de soluciones.

Para encontrar el valor de \(\alpha\) necesitamos reducir la matriz
ampliada de este sistema de ecuaciones de manera que encontremos la
última fila dependiendo de este escalar. Si \(\alpha\), para este
momento es igual a cero, entonces existen soluciones infinitas.

\[\begin{pmatrix} 1 & -1 & \alpha & -2 \\ 0 & 1 & 0 & 1 \\ 0 & 0 & (1 - \alpha^{2}) & .... \end{pmatrix}\]

Esta es la matriz reducida, y el valor \((1 - \alpha^{2})\) es lo que
nos interesa.

\[1 - \alpha^{2} = 0\]

Si \(\alpha\) tiene alguno de los valores del conjunto \(\set{-1, 1}\),
entonces el sistema tiene soluciones infinitas.

\subsubsection{Literal c)}\label{literal-c-3}

Suponga que existe una única solución para una a determinada, encuentre
la solución.

Asignemos de manera: \(\alpha = 2\). Con esto, calculemos las soluciones
mediante numpy.

\begin{Shaded}
\begin{Highlighting}[]

\NormalTok{A }\OperatorTok{=}\NormalTok{ [}
\NormalTok{    [}\DecValTok{1}\NormalTok{, }\OperatorTok{{-}}\DecValTok{1}\NormalTok{, }\DecValTok{2}\NormalTok{],}
\NormalTok{    [}\OperatorTok{{-}}\DecValTok{1}\NormalTok{, }\DecValTok{2}\NormalTok{, }\OperatorTok{{-}}\DecValTok{2}\NormalTok{],}
\NormalTok{    [}\DecValTok{2}\NormalTok{, }\DecValTok{1}\NormalTok{, }\DecValTok{1}\NormalTok{]}
\NormalTok{]}

\NormalTok{b }\OperatorTok{=}\NormalTok{ [}\OperatorTok{{-}}\DecValTok{2}\NormalTok{, }\DecValTok{3}\NormalTok{, }\DecValTok{2}\NormalTok{]}

\NormalTok{x }\OperatorTok{=}\NormalTok{ np.linalg.solve(A, b)}
\BuiltInTok{print}\NormalTok{(x)}
\end{Highlighting}
\end{Shaded}

\begin{verbatim}
[ 1.  1. -1.]
\end{verbatim}

Comprobamos que el sistema tiene solución.

\subsection{Ejercicio 6}\label{ejercicio-6}

Suponga que en un sistema biológico existen \(n\) especies de animales y
\(m\) fuentes de alimento. Si \(x_j\) representa la población de las
j-ésimas especies, para cada \(j = 1, ... , n\),;\(b_i\); representa el
suministro diario disponible del i-ésimo alimento \(a_{ij}\)𝑗 representa
la cantidad del i-ésimo alimento

\subsubsection{Literal a)}\label{literal-a-5}

Si:

\[\begin{pmatrix} 1 & 2 & 0 & 3 \\ 1 & 0 & 2 & 2 \\ 0 & 0 & 1 & 1 \end{pmatrix}\]

\(x = (x_j) = (1000, 500, 350, 400)\) y
\(b = (b_i) = (3500, 2700, 900)\). ¿Existe suficiente alimento para
satisfacer el promedio consumo diario? .

\begin{Shaded}
\begin{Highlighting}[]

\NormalTok{A }\OperatorTok{=}\NormalTok{ [}
\NormalTok{    [}\DecValTok{1}\NormalTok{, }\DecValTok{2}\NormalTok{, }\DecValTok{0}\NormalTok{, }\DecValTok{3}\NormalTok{],}
\NormalTok{    [}\DecValTok{1}\NormalTok{, }\DecValTok{0}\NormalTok{, }\DecValTok{2}\NormalTok{, }\DecValTok{2}\NormalTok{],}
\NormalTok{    [}\DecValTok{0}\NormalTok{, }\DecValTok{0}\NormalTok{, }\DecValTok{1}\NormalTok{, }\DecValTok{1}\NormalTok{]}
\NormalTok{]}

\NormalTok{x }\OperatorTok{=}\NormalTok{ [}\DecValTok{1000}\NormalTok{, }\DecValTok{500}\NormalTok{, }\DecValTok{350}\NormalTok{, }\DecValTok{400}\NormalTok{]}

\NormalTok{x }\OperatorTok{=}\NormalTok{ np.matmul(A, x)}
\BuiltInTok{print}\NormalTok{(x)}
\end{Highlighting}
\end{Shaded}

\begin{verbatim}
[3200 2500  750]
\end{verbatim}

Al parecer, el alimento fue suficiente para satisfacer a las poblaciones
de especies que conviven debido al suministro por donde se parte
(\([3500, 2700, 900]\)).

\subsubsection{Literal b)}\label{literal-b-5}

Cuál es el número máximo de animales de cada especie que se podría
agregar de forma individual al sistema con el suministro de alimento que
cumpla con el consumo?

\begin{Shaded}
\begin{Highlighting}[]

\NormalTok{A }\OperatorTok{=}\NormalTok{ [}
\NormalTok{    [}\DecValTok{1}\NormalTok{, }\DecValTok{2}\NormalTok{, }\DecValTok{0}\NormalTok{, }\DecValTok{3}\NormalTok{],}
\NormalTok{    [}\DecValTok{1}\NormalTok{, }\DecValTok{0}\NormalTok{, }\DecValTok{2}\NormalTok{, }\DecValTok{2}\NormalTok{],}
\NormalTok{    [}\DecValTok{0}\NormalTok{, }\DecValTok{0}\NormalTok{, }\DecValTok{1}\NormalTok{, }\DecValTok{1}\NormalTok{]}
\NormalTok{]}

\NormalTok{b }\OperatorTok{=}\NormalTok{ [}\DecValTok{3500}\NormalTok{, }\DecValTok{2700}\NormalTok{, }\DecValTok{900}\NormalTok{]}

\NormalTok{x }\OperatorTok{=}\NormalTok{ np.linalg.solve(A, b)}
\BuiltInTok{print}\NormalTok{(x)}
\end{Highlighting}
\end{Shaded}

\begin{verbatim}
LinAlgError: Last 2 dimensions of the array must be square
---------------------------------------------------------------------------
LinAlgError                               Traceback (most recent call last)
Cell In[68], line 9
      1 A = [
      2     [1, 2, 0, 3],
      3     [1, 0, 2, 2],
      4     [0, 0, 1, 1]
      5 ]
      7 b = [3500, 2700, 900]
----> 9 x = np.linalg.solve(A, b)
     10 print(x)

File ~\anaconda3\Lib\site-packages\numpy\linalg\linalg.py:396, in solve(a, b)
    394 a, _ = _makearray(a)
    395 _assert_stacked_2d(a)
--> 396 _assert_stacked_square(a)
    397 b, wrap = _makearray(b)
    398 t, result_t = _commonType(a, b)

File ~\anaconda3\Lib\site-packages\numpy\linalg\linalg.py:213, in _assert_stacked_square(*arrays)
    211 m, n = a.shape[-2:]
    212 if m != n:
--> 213     raise LinAlgError('Last 2 dimensions of the array must be square')

LinAlgError: Last 2 dimensions of the array must be square
\end{verbatim}

Creí que era de esta manera, pero no. Lo hice de manera analítica, que
como resultado, la especie 1 puede llegar hasta 1800 integrantes. La
especie 2, 3 y 4 dependen de ellas mismas. No se puede determinar de
forma exacta ya que hay muchas distribuciones posibles

\subsubsection{Literal c)}\label{literal-c-4}

Si la especie 1 se extingue, ¿qué cantidad de incremento individual de
las especies restantes se podría soportar?

Esto significa que\ldots{}

\begin{Shaded}
\begin{Highlighting}[]

\NormalTok{A }\OperatorTok{=}\NormalTok{ [}
\NormalTok{    [}\DecValTok{2}\NormalTok{, }\DecValTok{0}\NormalTok{, }\DecValTok{3}\NormalTok{],}
\NormalTok{    [}\DecValTok{0}\NormalTok{, }\DecValTok{2}\NormalTok{, }\DecValTok{2}\NormalTok{],}
\NormalTok{    [}\DecValTok{0}\NormalTok{, }\DecValTok{1}\NormalTok{, }\DecValTok{1}\NormalTok{]}
\NormalTok{]}

\NormalTok{b }\OperatorTok{=}\NormalTok{ [}\DecValTok{3500}\NormalTok{, }\DecValTok{2700}\NormalTok{, }\DecValTok{900}\NormalTok{]}

\NormalTok{x }\OperatorTok{=}\NormalTok{ np.linalg.solve(A, b)}
\BuiltInTok{print}\NormalTok{(x)}
\end{Highlighting}
\end{Shaded}

\begin{verbatim}
LinAlgError: Singular matrix
---------------------------------------------------------------------------
LinAlgError                               Traceback (most recent call last)
Cell In[71], line 9
      1 A = [
      2     [2, 0, 3],
      3     [0, 2, 2],
      4     [0, 1, 1]
      5 ]
      7 b = [3500, 2700, 900]
----> 9 x = np.linalg.solve(A, b)
     10 print(x)

File ~\anaconda3\Lib\site-packages\numpy\linalg\linalg.py:409, in solve(a, b)
    407 signature = 'DD->D' if isComplexType(t) else 'dd->d'
    408 extobj = get_linalg_error_extobj(_raise_linalgerror_singular)
--> 409 r = gufunc(a, b, signature=signature, extobj=extobj)
    411 return wrap(r.astype(result_t, copy=False))

File ~\anaconda3\Lib\site-packages\numpy\linalg\linalg.py:112, in _raise_linalgerror_singular(err, flag)
    111 def _raise_linalgerror_singular(err, flag):
--> 112     raise LinAlgError("Singular matrix")

LinAlgError: Singular matrix
\end{verbatim}

Para este caso no parece existir una distribución exacta.

\subsubsection{Literal d)}\label{literal-d-3}

Si la especie 2 se extingue, ¿qué cantidad de incremento individual de
las especies restantes se podría soportar?

\begin{Shaded}
\begin{Highlighting}[]

\NormalTok{A }\OperatorTok{=}\NormalTok{ [}
\NormalTok{    [}\DecValTok{1}\NormalTok{, }\DecValTok{0}\NormalTok{, }\DecValTok{3}\NormalTok{],}
\NormalTok{    [}\DecValTok{1}\NormalTok{, }\DecValTok{2}\NormalTok{, }\DecValTok{2}\NormalTok{],}
\NormalTok{    [}\DecValTok{0}\NormalTok{, }\DecValTok{1}\NormalTok{, }\DecValTok{1}\NormalTok{]}
\NormalTok{]}

\NormalTok{b }\OperatorTok{=}\NormalTok{ [}\DecValTok{3500}\NormalTok{, }\DecValTok{2700}\NormalTok{, }\DecValTok{900}\NormalTok{]}

\NormalTok{x }\OperatorTok{=}\NormalTok{ np.linalg.solve(A, b)}
\BuiltInTok{print}\NormalTok{(x)}
\end{Highlighting}
\end{Shaded}

\begin{verbatim}
[900.          33.33333333 866.66666667]
\end{verbatim}

En este caso se determinó la población que pueden tener cada una de las
especies (Especie 1: 900, especie 2: 33, especie 3: 866)




\end{document}
